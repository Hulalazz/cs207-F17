\documentclass[11pt]{article}

\usepackage{amsmath}
\usepackage[margin=1in]{geometry}

\begin{document}

  \begin{abstract}
    The purpose of this document is to summarize what worked and what did not
work in CS207-Systems development of Computational Science.  The goal is not to
plan out the following semester.  This document should simply contain helpful
notes so that planning the next iteration of the course is easier.
  \end{abstract}

  \section{Fall 2017}
    \subsection{Course Content}
    The course content was mostly well-received by the students.  The consensus
was that there was too much of a focus on basic Python.  We spent the first
month or so on very simple Python assignments and concepts.  Some students
really appreciated this; especially the non-negligible portion who had never
coded before or had very limited coding experience.  However, this made it
particularly difficult to cover some more interesting software engineering
practices such as pipelines.  I also probably spent too much time on OOP.
Students got really bored with this.  The database topics were really
well-received.  Students like the execution (spending class time working on
\texttt{SQL}) and that they got to do some stuff with databases.  Should
definitely keep that topic.  Maybe expand/extend it?  Should probably get a
guest lecture on databases.  Niv (if he's still around) said he'd be willing to
do it since he did it one other year for Rahul.

    Some things to think about for the next iteration:
    \begin{itemize}
      \item Coding bootcamp.  Admitted students should take a self-assessment on
their coding abilities.  If the pass, then they can take CS207 in the fall
semester.  If they do not pass, then they must take a 10-day coding bootcamp in
order to enroll in CS207 for the fall semester.
      \item Less focus on how Python works under the hood.  Some of this content
can be moved to homework assignments.
      \item Several students mentioned that they wanted to learn how to use the
\texttt{requests} library.  Maybe I can work that in somehow.
    \end{itemize}
    \subsection{Course Structure}
      \subsubsection{Meeting Times}
        The meeting times seemed to work well.  We met twice a week (T and Th)
for 1.5 hours each time.  This allowed some lecture time as well as some
in-class work time.  A future iteration may benefit from lab sessions.  This
were not necessary this past semester because the course was not very
time-consuming for students.
      \subsubsection{Homeworks}
        There ended up being 10 homework assignments in total.  This was a
pretty good number.  I dropped the lowest grade, but that turned out to not be
necessary since everyone did so well.  The early assignments were far too easy.
This will be changed in the next iteration because we will jump into advanced
topics much earlier.  In that case, it will still be a good idea to drop the
lowest homework grade.

        I should figure out a better way to grade the assignments.  One of the
TFs (Charles) wrote his own test suite for the problems that he graded.  This is
a cool idea, but punishes students for small mistakes and does not reward
effort.  One way around this would be to come up with an alternate grading
scheme.  Right now, we grade each assignment out of 10 points.  Such a scheme
may not offer enough flexibility.  A more flexibile grading scheme may allow us
to reward effort more while awarding students who did a really good job.  I
think I will still ask the TFs to write their own problems.  However, I need to
do a better job of checking their problems.  Does Pavlos have his TFs write
their own assignments or does he write his own?

        Other thoughts on HWs:
        \begin{itemize}
          \item Bonus problems?  We could offer a challenge problem so advanced
students can still be challenged.  Less advanced students wouldn't be punished
for not doing such problems.
          \item Provide homework solutions?  I think students would really love
this.  We need to make sure the solutions are thorough and easy to follow.  This
would be one way to ensure that the TFs write good problems as well as to make
sure I check their problems and solutions.  Would help me feel more involved in
the process.
          \item Homework due date:  Midnight seemed to work well.  Give them a
week for each assignment with the exception of particularly challenging
assignments.  Make sure you give them enough time to visit all office hours.
Many of them probably won't start the assignment until the day before it's due.
In such a case, having homework due on Monday morning at midnight is probably
not going to work great since many of them won't start working until the
weekend.  They won't be able to go to office hours and then they might
complain.  Wednesday or Thursday due dates seem to work well depending on when
the course lectures are scheduled.
        \end{itemize}
      \subsubsection{Lecture Exercises}
        These didn't work as great as I had hoped.  They ended up being in a
grey zone between regular lectures and a full flipped classroom.  Students
didn't universally hate them, but they didn't really adore them either.
Moreover, I didn't even get to have every group come to the front of the class
to present.  When they did present, one student did pretty much all the
presenting and the other mainly stood idly by.

        It may be a really good idea to have a few flipped classrooms scattered
throughout the semester.  These are expensive to create, but the students
absolutely love them.  Maybe we can have around 3-5 flipped classrooms on
particular topics.  Or, another idea, is to have one flipped classroom per major
topic.  For example, one flipped classroom for pipelines, one for algorithms and
data structures, etc.  Not exactly sure how this will play with any lab
sessions.
    \subsection{Final Project}
      The final project went pretty well.  It may have been a bit on the easy
side, but I think this was good for some students.  The group selection worked
out really great.  Manually pairing students with each other so that strong
students were paired with inexperienced students was excellent.  Should try this
again.  Not every student was crazy about the chemical kinetics project.
However, I also received a lot of really great feedback on the project.  

      Pavlos is pushing for a different project for next year.  Maybe we'll do a
class project.  Or maybe we'll still do groups.  The project will probably be
tailored more towards data science rather than computational science.  Perhaps
the students can develop something for \texttt{scikit-learn} in \texttt{Python}.
Still thinking about how to do this.
    \subsection{TF Management}
      The TFs were really great.  It is very important to get good TFs!  I
should probably have asked them to do more work.  Maybe I can have them give
guest lectures in the future?  This would be good for their career and also help
me assess them.  Having them write the homework assignments worked pretty well,
but I should probably have been more involved in that process.  Somehow I feel
that I should have given the TFs more work to do.  I'm not sure how to do that
though.  Just keep this in mind when planning out the next iteration of the
course.

      Another thing to think about with the TFs is their choice of office hours.
I need to come up with some parameters on how they should choose their office
hours.  For example, they can't have their office hours immediately after the
homework is meant to be turned in.  They also shouldn't have their office hours
too late at night.  These were not attended very well.  Of course, with more
challenging course content things might change a bit.

\end{document}
